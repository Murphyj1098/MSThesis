\chapter{Intelligent Router Source Code}
\label{appendixb}
\section{Control Script}
\begin{minted}[fontsize=\small, linenos, breaklines]{python}
#!/usr/local/bin/python3.8
import time
import datetime
import logging

import classify
import allocate

if __name__ == '__main__':

    date = datetime.date.today()
    logFileName = "./Logs/{today}.log".format(today=date)

    # Setup logging
    logging.basicConfig(filename=logFileName, filemode='a', level=logging.DEBUG,
                        format='%(asctime)s - %(levelname)s:%(message)s', datefmt='%m/%d/%Y %I:%M:%S %p')

    nextStart = time.monotonic()
    delta = 7.5  # 7.5 seconds between runs (time to perform all data collection and processing)

    for i in range(8):
        nextStart = nextStart + delta

        # Classify code here
        classData = classify.main()

        # Allocation code here
        allocate.main(classData)

        # Wait until next classify time
        if i != 7:
            while nextStart > time.monotonic():
                time.sleep(0.1)
\end{minted}

\section{Classification}
\begin{minted}[fontsize=\small, linenos, breaklines]{python}
#!/usr/local/bin/python3.8
import datetime
import logging
import re
import subprocess


# Get bandwidth data from iftop and parse
def flow():
    # Run iftop
    # Arguments: -t, text mode (remove ncurses)
    #            -c <file>, configuration input file
    #            -s #, measure for # seconds
    #            -i <network interface>, interface to listen on
    #            -L #, number of lines to display
    #
    # Redirect stderr to /dev/null
    # Take stdout output and split each line into list
    iftop = "iftop -t -c .iftoprc -s 3 -L 35 -i ens18"  # (FIXME: Change network interface to correct value)
    proc_out = subprocess.run(args=iftop, shell=True, universal_newlines=True,
                              stdout=subprocess.PIPE, stderr=subprocess.DEVNULL)
    top_list = proc_out.stdout.split("\n")

    # Trim list to only contain relevant lines of data (data with per IP bandwidth)
    data_list = []

    for i in range(len(top_list)):
        if re.search('^ {1,3}[0-9]', top_list[i]):
            data_list.append(top_list[i])
            data_list.append(top_list[i + 1])
            i + 1

    # Count = 2 * number of hosts
    # Two lines per host (one upload, one download)
    count = len(data_list)

    # If list is empty, no data to work on
    if count < 2:
        return -1

    # Dictionary to hold host information, in-coming, and out-going traffic
    global host_dict
    global host_list

    host_dict = {}
    host_list = []

    #
    # For each host upload/download pair, extract information into format below
    #
    #     Host     |  Up Rate | Down Rate
    #   <ip_addr>  |   Mbps   |  Mbps
    #   <ip_addr>  |   Mbps   |  Mbps
    #      ""      |    ""    |   ""
    for i in range(int(count / 2)):
        down_list = data_list[i * 2].split(" ")
        up_list = data_list[(i * 2) + 1].split(" ")

        while '' in up_list:
            up_list.remove('')

        while '' in down_list:
            down_list.remove('')

        host_ip = up_list[0]
        up_rate = up_list[2]
        down_rate = down_list[3]

        # Standardize units
        up_rate = unit(up_rate)
        down_rate = unit(down_rate)

        # Store data
        host_data = [up_rate, down_rate]
        host_dict[host_ip] = host_data
        host_list.append(host_ip)


# Classify each host's priority
def priority():

    global prio_dict
    prio_dict = {}

    prio = 5

    for ip in host_list:
        bandwidth = host_dict[ip]

        if bandwidth[0] < 200.0 and bandwidth[1] < 5000.0:
            prio = 0
        elif bandwidth[0] > 200.0 and bandwidth[1] < 5000.0:
            prio = 1
        elif bandwidth[0] < 200.0 and bandwidth[1] > 5000.0:
            prio = 2
        elif bandwidth[0] > 200.0 and bandwidth[1] > 5000.0:
            prio = 3

        prio_dict[ip] = prio

        logging.info(" CLASSIFY -- Host: %s; Upload: %.2fKbps, Download: %.2fKbps, Priority: %d", ip, bandwidth[0], bandwidth[1], prio)


# Strip unit, standardize to Kbps
def unit(measure):
    if "Mb" in measure:
        ret = float(measure.strip("Mb")) * 1024
        return ret
    elif "Kb" in measure:
        ret = float(measure.strip("Kb"))
        return ret
    elif "b" in measure:
        ret = float(measure.strip("b"))
        return ret


def main():

    outputDict = {}

    if flow() != -1:
        priority()
        return host_dict

    else:
        logging.info(" CLASSIFY -- No data found")


if __name__ == '__main__':

    date = datetime.date.today()
    logFileName = "./Logs/{today}.log".format(today=date)

    # Setup logging
    logging.basicConfig(filename=logFileName, filemode='a', level=logging.DEBUG,
                        format='%(asctime)s - %(levelname)s:%(message)s', datefmt='%m/%d/%Y %I:%M:%S %p')

    main()
\end{minted}

\section{Allocation}
\begin{minted}[fontsize=\small, linenos, breaklines]{python}
#!/usr/local/bin/python3.8
import csv
import datetime
import logging
import os
import xml.etree.ElementTree as ET


# Standard amount by which bandwidth should be raised or lowered (Kbps)
incrementAmount = 2500
decrementAmount = 1000

# Individual Host max BW and min BW (Kbps)
minHost = 7500
maxHost = 100000

# Max bandwidth (Kbps)
maxNetworkBandwidth = 25000 * 30

configFile = 'config.xml'       # pfSense XML setting file (FIXME: Restore this path to '/conf/config.xml')
classifyFile = 'classData.csv'  # classify script output file


def readClassifyData():

    # Read data from classification script
    # Dictionary {Key: Value} where Key is <ip address> and Value is (Upload, Download)
    classifyData = {}

    with open(classifyFile) as csvFile:
        csvReader = csv.reader(csvFile)
        for row in csvReader:
            classifyData[row[0]] = (row[1], row[2])

    return classifyData


def readXML():

    # Get current download bandwidth allocations from limiters

    tree = ET.parse(configFile)
    root = tree.getroot()

    currBW = {}

    for queue in root.findall('./dnshaper/queue'):
        ip_end = queue[0].text
        if len(ip_end) > 5:  # Catch limiters not named _# (where # is last octet of IP address)
            continue
        ip = "192.168.50.%s" % ip_end[1:]  # (FIXME: Restore start of IP address to appropriate value)
        bw = queue[5][0][0].text

        currBW[ip] = bw

    return currBW


def genBWList(currentAllocation, classifyData):

    # Generate dictionary of IP : New Bandwidth Amount (In Kbps)
    newBWList = {}

    for key in currentAllocation:

        if classifyData is None or key not in classifyData:
            # Assume no reading means no usage
            if int(currentAllocation[key]) > minHost:
                newBWList[key] = str(int(currentAllocation[key]) - decrementAmount)
                logging.info(" ALLOCATE -- Host: %s; Decrease cap, New Cap: %s", key, newBWList[key])
            continue

        downloadVal = classifyData[key][1]

        if downloadVal > (int(currentAllocation[key]) * 0.95) and int(currentAllocation[key]) < maxHost:
            newBWList[key] = str(int(currentAllocation[key]) + incrementAmount)
            logging.info(" ALLOCATE -- Host: %s; Increase cap, New Cap: %s", key, newBWList[key])
        elif downloadVal < (int(currentAllocation[key]) * 0.50) and int(currentAllocation[key]) > minHost:
            newBWList[key] = str(int(currentAllocation[key]) - decrementAmount)
            logging.info(" ALLOCATE -- Host: %s; Decrease cap, New Cap: %s", key, newBWList[key])
        else:
            newBWList[key] = currentAllocation[key]
            logging.info(" ALLOCATE -- Host: %s; No change, New Cap: %s", key, newBWList[key])

    return newBWList


def writeXML(newBW):

    # 1. Get new bandwidth amounts as input arg (dictionary)
    # 2. Parse /conf/config.xml and find each queue's IP
    # 3. Use IP as key in dictionary to get new bandwidth value
    # 4. Write out new XML file

    tree = ET.parse(configFile)
    root = tree.getroot()

    for queue in root.findall('./dnshaper/queue'):
        ip_end = queue[0].text
        if len(ip_end) > 5:  # Catch limiters not named _# (where # is last octet of IP address)
            continue
        ip = "192.168.50.%s" % ip_end[1:]  # (FIXME: Restore start of IP address to appropriate value)
        try:
            queue[5][0][0].text = newBW[ip]
        except KeyError:  # Skip queue if corresponding IP is not in dictionary
            continue

    tree.write(configFile)

    return 0


def reloadFirewall():

    # 1. Remove /tmp/config.cache (Reloads config file)
    # 2. Run /etc/rc.filter_configure (Reloads pfsense firewall)

    os.system('rm /tmp/config.cache')
    os.system('/etc/rc.filter_configure')

    return 0


def main(classData):

    # 1. Get and store current bandwidth allocations
    # 3. Generate list of new bandwidth allocations
    # 4. Write new allocation parameters out to XML File
    # 5. Reload firewall

    currAllots = readXML()
    newBW = genBWList(currAllots, classData)
    writeXML(newBW)
    reloadFirewall()

    return 0


if __name__ == '__main__':

    date = datetime.date.today()
    logFileName = "./Logs/{today}.log".format(today=date)

    # Setup logging
    logging.basicConfig(filename=logFileName, filemode='a', level=logging.DEBUG,
                        format='%(asctime)s - %(levelname)s:%(message)s', datefmt='%m/%d/%Y %I:%M:%S %p')

    main({})
\end{minted}
