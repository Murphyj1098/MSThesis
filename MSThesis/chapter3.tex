\chapter{Hybrid Wireless Rural Broadband Networks}
\label{chapter3}

    % This is the first use-case for EMANE
    % System was designed as an easily implementable, low-cost solution for communities that are too small for a dedicated fiber backhaul, but still close enough to the fiber for a wireless solution

\section{What was implemented?}
    % Hardware testbed utilized a mmWave backhaul and Ubiquiti LTU radios for the distribution network
    % mmWave and LTU were abstracted using rfPipe in EMANE
    % Allowed for testing throughput and bandwidth characteristics with multiple hosts connected as well as range using simple pathloss models

\subsection{OVERCOME Testbed}
    % Topology
        % (FIGURE\_OVERCOMETopology)
        % Two sets of mmWave backhauls
        % Aggregate at a router in the middle of town
        % Split into 3 radios for distribution to ~30 houses

    % Technologies used
        % Ubiquiti AirFiber, Ubiquiti Edge Switches, Netgate pfSense Router, Ubiquiti LTU Rockets
        % (turns into)
        % mmWave rfPipe model, virtual bridge acting as a switch, virtualized pfSense router in VirtualBox, LTU rfPipe model

\subsection{ZoomTEL Testbed}
    % Topology
        % (FIGURE\_ZoomTELTopology)
        % Two-hop mmWave backhaul
        % LTE-U distribution network at each hop
        % Distribute to hosts centered around each hop

    % Technologies used
        % Siklu EtherHaul, Ubiquiti network switches, Ubiquiti LTU Rockets
        % (turns into_
        % mmWave rfPipe model, virtual bridge acting as a switch, LTU rfPipe model


\section{How was it implemented?}
    % Creating mmWave and LTU in EMANE
        % Use rfPipe to mimic characteristics of mmWave and LTU
            % Bandwidth, delay, throughput, frequency of interest, distance between radios, antenna RX and TX gains

    % Interfacing with a real pfSense router in EMANE
        % Use EMANE transport plugins to delivery and receive real TCP/IP packets from a virtualized router
        % Performance implications of using a virtualized router vs. the real hardware router?
            % (No, there was such limited traffic that the hardware router was never pushed beyond a few percent usage, and the virtual router never bottlenecked)

    % How were things like throughput and latency tested?
        % In EMANE?
            % Tools like iperf3 were used for generating traffic
            % Test probes in EMANE were used for monitoring the "behavior" of the physical channel
        % In the hardware testbed?
            % Ookla's speed test was used for collecting data through homes
            % pfSense router monitored and recorded total data usage via the "bandwidthD" utility
            % iftop was used (as apart of the intelligent router) and recorded average bandwidth usage during certain testing periods


\section{Why was it implemented this way?}
    % What were the important features to capture in EMANE
        % Throughput
        % Latency
        % Estimated pathloss (and packet loss) over expected hardware distances
        % Anything else?


\section{Compare EMANE results to real testbed?}
    % How similar are the results from EMANE to the heardware testbed?

\section{Chapter Summary}

