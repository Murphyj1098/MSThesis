\chapter{Networking Software Development Environment}
\label{chapter4}
    % Broadband in rural situations is limited due to reasons explained in Chapter 3
    % In addition to creating and testing new topologies, other methods can be used to maximize the deliverable bandwidth
    % In this case, giving users bandwidth in an intelligent manner can allow the network to perform better
    % Utilizes the OVERCOME testbed from Chapter 3
The second scenario EMANE was used for is the development of an intelligent resource allocation program.
The goal of the network topology presented in \ref{chapter3} was to deliver greater amounts of connectivity to underserved rural communities.
Introducing new networking hardware is one way of achieving this goal, but another potential for increasing the usability of the Internet is to better allocate resources in an intelligent manner.
By using the EMANE testbed developed in the previous chapter, an accurate environment for developing this software can be created.
Testing inside an emulated EMANE environment allows the software to still act on an accurate network without needing to be deployed to users.
Developing in a production environment could have potentially several issues, such as violating the privacy concerns of users, or diminishing the quality of service a user experiences.
This chapter will present the initial creation of a tool to achieve the goal of more intelligently distributing network resources.

\section{Intelligent Method of Bandwidth Distribution}
    % A heuristic based router that intelligently allocates bandwidth to hosts
    % Based on the behavior of uploads and downloads


\section{Implementation Methodology}
    % (FIGURE\_classify+allocate\_flowcharts)
    % Want to classify each host's current upload/download usage
        % Used tool: iftop to get average usage over 2 seconds, record usage every 6 seconds (10 times per minute)
        % (FIGURE\_iftop)
    % Use this classification to create priority tiers that influence how users get more bandwidth/bandwidth taken away
        % What is the rationale behind these priorities (study on what types of behaviors use what types of bandwidth)
    % Create these modifications to move bandwidth caps up and down via pfSense limiters


\section{Why was it implemented this way?}
    % Heuristic behavior was selected over a machine learning model
        % Why?
    % System privacy concerns
        % Did not want to classify based on the traffic type and what the user was doing
        % To create a fair algorithm can not define one type of traffic as "more important"
            % Is a Zoom call more important than a Netflix stream?
            % Business vs. Pleasure BUT Zoom is not always business and Netflix is not always leisure
    % Designed to be more system agnostic (tested on a pfSense Router)
        % Utilizes pfSense-specific limiters but the idea of setting a limit is very universal
        % Lighter weight than a ML algorithm likely would be, can be deployed closer to the last mile since it is lighter-weight

        
\section{Effectiveness of the Program}
    % Only deployed for an initial iteration due to network issues
        % No revisions made, only tested the most basic behavior
    % Not enough traffic/users on the network for meaningful data
        % At no point during any test was total network utilization above 50%, can't take bandwidth from users to give to others if there is always a pool
    % Ways to modify/fix the experiment and test environment to create a meaningful test


\section{Chapter Summary}

