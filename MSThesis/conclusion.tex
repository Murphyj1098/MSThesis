\chapter{Conclusion}
\label{conclusion}
\section{Evaluating EMANE and Network Emulation}
%% Critical discussion of limitations of emulating networks (as done in the thesis)
This thesis began by proposing network emulation as a solution to the difficulties present in developing and testing communication networks.
Chapter~\ref{chapter2} introduced several issues that often occur in network testing and development, including the expensive nature of hardware testbeds, the difficulties in creating an accurate enough testbed, and the time required to test multiple network topologies and configurations.
The question now remains if emulation is able to address these issues, and the remainder of this section will attempt to answer this.\par
% Limitations of emulation include:
    % Performance problems (unstable performance, could help to explore a distributed approach to EMANE)
        % Chapter 3 showed a 200Mbps throughput with EMANE, but was often unstable (during certain tests throughput would be real small)
        % Distribution of nodes helps with compute limitations, but the emulation is then limited by the link connecting them all
            % This could add expensives if something like 10 gig direct attach copper is required
    % Difficulty of understanding the tool/availability of preconfigured tools
A few limitations with EMANE were discovered in testing that must be considered when determining if EMANE is an appropriate tool.
The first limitation relates to the performance of emulated networks.
All the networks emulated in this thesis were able to match the expected network performance. This was primarily due to the careful configuration of the tool.
Every design decision was made while considering the performance impacts they would cause, and even with these considerations, initial configurations had instability.
These instabilities are caused by the computational load generated when all packets in a network have to be handled by a single kernel, a problem unique to emulation.
The system was eventually made stable by ensuring enough computational resources were available for the task.
Nevertheless, in larger testbeds, a single server may not be sufficient, and a distributed approach to emulation may need to be considered.
The other primary limitation found with emulation and EMANE was the initial difficulty with using the tool.
Emulation was suggested as an easier alternative to other testing methods, and it still could be easier provided enough documentation and tutorials are created.
The difficulty with understanding how to use EMANE was specifically attributed to the tool's small user base.
As more researchers use the tool, it becomes easier to find help.
The same can be said for the difficulty in using network simulation and networking hardware.
Every tool will have an initial period of difficulty, especially if adequate documentation does not exist, and for this reason, the difficulties experienced with EMANE should not invalidate it.\par
% Use specific experience from research to prove/disprove:
    % Emulation lowers the difficulty of testing (especially in hardware)
        % Learning the tool can be difficult 
    % Emulation reduces costs
        % EMANE free -- OVERCOME (Hardware) $>15000
    % Allow for limited testing of hardware (without a whole hardware network)
Despite these limitations, we still found emulation works as a good tool for testing networks.
Emulation primarily lowers the cost of testing, especially when using an open-source tool like EMANE that costs nothing to use.
For comparison, the hardware alone in the testbed built in Chapter~\ref{chapter3} cost over \$15,000 USD, excluding costs for labor or permits needed to deploy the network.
This significant difference in cost is a strong argument for simulation and emulation.
Another issue proposed was the difficulty in creating an accurate testbed.
Our experiments showed that in addition to being flexible, EMANE is also able to model several factors of a communications network accurately.
In addition to writing models that provide accuracy, hardware can be interfaced with the tool (as shown in Chapter~\ref{chapter4}) or other software tools can be integrated to provide accuracy for externally controlled factors (as seen in Chapter~\ref{chapter5}).
This desire for an accurate representation of the physical channel provides emulation with an advantage over simulation, as being able to use hardware for specific segments can ensure perfect accuracy. 
As long as the tool is used with care to ensure accuracy (as is the case with most testing tools), network emulation can provide an ideal testing environment for communication networks.

\section{Limitations of Network Emulation}
The previous section discussed a few limitations found with EMANE during testing.
This section aims to provide concluding thoughts on network emulation as a whole and some limitations and shortcomings discovered.\par
One of the biggest issues with network emulation, is the aforementioned performance limitations.
These limitations create issues with running large-scale testbeds, but more importantly can create inaccuracies in the results.
The delay throughput of a channel are carefully configured, but these values do not take into account added delays caused by the kernel.
Packets switching through the kernel are assumed to have a small enough delay and high enough throughput that they can be ignored.
This behavior is acceptable for small testbeds that do not push the limits of the host machine, but in larger testbeds where the host machine is pushed to its limit, the impact of the kernel may affect results (especially in very high speed, low latency networks).
This problem does not disappear when moving to a distributed layout.
The distribution of work to other machines fixes the issue with the kernel being overloaded, but introduces the new issue of the connecting fabric between servers imparting its own communication behaviors.
If two machines are directly attached, the issue is minimal, but if a network switch (or worse a router) is introduced, the latency increases.
Sufficiently fast hardware connecting emulation hosts can be purchased, but this is an expense that raises the total cost of emulation testing. \par
An additional limitation of network emulation is the amount of configuration available.
High configurability is beneficial to some degree, but if too many attributes need to be set, it becomes more likely that data that is not readily available is required.
Communication technologies that have been extensively studied, like the IEEE802.11 family, has plenty of data available to reference, but if the technology being emulated is new or less commonly used, a hardware testbed may be needed to simply configure the emulation testbed.
At that point, the built hardware testbed could just be used for testing instead eliminating the need for emulation in the first place.\par
The third major limitation to network emulation is the ability for randomness to affect results. 
One of the flaws with hardware testbeds is the random nature of the operation environment, affecting results and making it more difficult to replicate issues.
Network emulation reduces some amount of randomness present in the testbed, but the presence of real network packets (both being operated on and generated) creates some inherent randomness.
If the host machine is running a process in the background unknown to the operator, this can create unexpected behavior as less resources are available to the emulator.
All of these limitations must be considered when selecting network emulation as the testing method of choice, and with careful consideration the limitations can either be deemed irrelevant or can be accounted for.

\section{Research Outcomes}
Overall, this project was able to achieve its primary goal of evaluating and understanding the EMANE tool.
Several scenarios were presented where EMANE was shown to be successful.
The first use-case of rural broadband testing, showed that the basic rfPipe radio model can be configured to match the characteristics of a hardware testbed.
This scenario does carry the warning that just because the characteristics of the network are the same, not every model in EMANE is validated, and care must be taken to ensure the emulator is configured right.
The rfPipe model is simple enough that the throughput and latency matching is enough to use it in basic testing, but advance models mimicking more complex behaviors must be tested further.
The second use-case shows that EMANE provides a good environment for testing and developing networking software.
Care must be taken when selecting and modeling test traffic, but EMANE provides many avenues for creating and using test data.
The third use case shows that EMANE is able to work with other tools to create a more complete system, with the timescale differences between software types identified as the major obstacle to be considered.

\section{Future Work}
    % Create more accurate and refined models for EMANE that can then be pluggable into testbeds
        % Once these models are created, more advance wireless emulation can be done with similar ease thanks to the plug and play nature of EMANE
        % These models would will need to be vigorously tested to ensure the validity of the results
    % Further testing of the intelligent router program in more appropriate conditions
    % Further integrations between ARGoS and EMANE to expand functionality and optimize time consumed
While each stage of this thesis had successes, there are still several pieces of work that remain uncompleted. The following list outlines this future work:
\begin{itemize}
    \item Conducting extensive validation of the models included within EMANE. EMANE is not as widely used as many other network testing tools, and as such has been studied less regarding its validity. If more advanced models are to be used in EMANE, it should be ensured they are accurate enough for the advanced use cases.
    \item Creating additional wireless models for EMANE. By default, EMANE only has four radio models accessible, with one being the generic model used in this thesis and another being used for testing of EMANE itself. If EMANE is to be more widely used, it needs to be able to model more modern waveforms and technologies.
    \item Additional testing and development of the intelligent router software. Due to the issues described in Chapter~\ref{chapter2} only one initial test was conducted with non-ideal conditions. Further testing should be conducted to evaluate the tool in ideal conditions
    \item Further integration and optimizations between EMANE and ARGoS. Finding ways to reduce the time EMANE adds onto the total simulation, while also adding functionality to allow for modeling of more communication factors like latency or advanced routing techniques would allow the integration to be even more useful.
\end{itemize}