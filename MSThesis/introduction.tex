\chapter{Introduction}
\label{ch:introduction}
\section{Motivation}
As the need Internet and wireless communications grow, more solutions for deploying networks are being conceived. 
Networks are expensive to deploy and test, especially in rural environments where hybrid combinations of wireless technologies must be used and when using special networks (MANET) % Statistics on expensive network deployment (especially for testing and research)
This creates a need for a low-cost, easy method to do initial testing on networks and network technologies to validate viability before spending money on hardware deployments and testing
Use network simulation and emulation for preliminary testing as it can require little to no hardware, can be conducted in a lab, and costs less than building physical networks for each new experiment


\section{State of the Art}
Many software and combination software-hardware platforms exist for testing networks.
ns-3, NetSim, OPNET (now owned and operated by Riverbed)


\section{Current Issues}
Why are these simulators not as good?
\begin{itemize}
    \item Many are not free or open-source (expensive to use and possibly not as customizable) % NetSim, OPNET both require payment
    \item Can be complex to set up % NS-3 is not GUI based, instead requires knowledge of C++ or Python; EMANE is also not GUI based but can be paired with CORE for a GUI in simple experiments (configuration files are also just XML)
    \item Often only focus on network layer and abstract MAC/PHY layer, OR model the MAC/PHY layer but does not allow for integration with network software and protocols % MATLAB can model physical channel parameters, but can not model the whole network
\end{itemize}


\section{Thesis Contribution}
\begin{itemize}
    \item EMANE is proposed as a valuable testing tool that addresses issues with other similar networking simulation tools. An overview of installing and using the tool is provided.
    \item Develop an initial simple program for maximizing bandwidth in constrained networks and showed how EMANE can be used as a valuable development environment.
    \item Basic integration between the robot swarm simulator, ARGoS, and EMANE allowing for more complex communications simulation between robots.
\end{itemize}


\section{Thesis Organization}
The remainder of this thesis is organized as follows:
Chapter 2 presents an overview of the network emulator EMANE and the motivation behind the selection of this tool. An overview of the how to use EMANE and its subsystems is presented.
Three different use-cases for the EMANE tool are considered to evaluate the effectiveness of the tool.
Chapter 3 proposes the first use case for testing with EMANE, testing rural broadband deployments. Two similar network topologies are proposed and tested with the help of EMANE.
Chapter 4 explores a second use case for utilizing EMANE, development of networking technologies and systems. In this case a program for more intelligent allocated limited bandwidth is developed.
Chapter 5 finally details a third use case for EMANE, integrating with other simulation tools to provide more accurate communication models.
The paper is concluded with a summary of work completed and recommendations for future work in Chapter 6.
